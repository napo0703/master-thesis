\chapter{関連研究}
\label{chap:relevant}

本章では、情報ダッシュボードに関する研究と
沢山の人がいる状況での計算機を用いたコミュニケーションの支援に関する研究を紹介する。

\newpage

\section{情報ダッシュボード}

情報ダッシュボードのデザイン\cite{few}に関する研究は多くないが,
表示するべき情報を選択する手法\cite{Jones:2015:ECI:2800835.2800963}や,
セルの自動配置手法\cite{Hertzog:2015:BSP:2678025.2701383}などの
研究が存在する.
『わかるらんど』のダッシュボードは
単純な形状のセルを指定どおりに並べているだけであるが,
よりわかりやすい表示のための配置手法の検討は意義があると思われる.

\section{沢山の人がいる状況での計算機を用いたコミュニケーションの支援}
会議での議論を促進するために
「On-Air Forum」\cite{nishida2011},
「Lock-on-Chat」\cite{nishida2006}
など様々なチャットシステムが提案されているが,
このようなチャットシステムのほとんどは
タイムライン型式で表示が行なわれるようになっており,
情報ダッシュボードのような型式で感情や意見など書き込んで
一覧できるチャットシステムは存在しない.

近年,
消極的な人間でも会議の議論などに参加しやすくするための研究が
消極性研究会(SIGSHY: Special Interest Group on Shyness and Hesitation around You)
というグループなどを中心に盛んになってきているが\cite{kurihara2016}\cite{nishida2011},
\ref{nerai}で述べたように,
『わかるらんど』もこのような方向性の支援に利用できる.
