\chapter{議論}
\label{chap:discussion}

本章では,第\ref{chap:experiment}章での結果をもとに,
『わかるらんど』システムの有効性と意義について議論を行う.

\newpage

\section{リアルタイムコミュニケーションとしての『わかるらんど』}

WISS2016の会期を通して,発表内容に関する議論はOn-Air Forum,
「笑」や「なるほど」などの相槌は『わかるらんど』に
それぞれ投稿するという棲み分けができており,
On-Air Forumに「笑」や「なるほど」などの相槌のような発言は全く見られなかった.
これは,相槌のような発言がチャットの議論空間を侵害するという意識が
参加者の中にあったためであると思われる.

On-Air Forumには発言を「Good」「Bad」で評価する反応ボタン機能があり,
WISS2009のOn-Air Forumの実証実験\cite{nishida2011}によると,
匿名で利用できるこの機能は投稿をせず見ているだけの人にもよく利用されていた.
『わかるらんど』のWISS2016での実証実験からも,
学生,教員,企業の研究者など様々なバックグラウンドの人が入り混じった状況で
匿名で意思表示ができることは有意義であると考えられる.


\section{『わかるらんど』に表示可能な情報の数}

WISS2016での『わかるらんど』の利用者は71人であったが,
この人数の投稿をダッシュボードに表示しても問題なく
利用者の投稿を把握することができた.
『わかるらんど』は実装上は表示するセルの数には制限を設けておらず,
いくつでもセルを追加することが可能であるが,
300個を超えると一般的なノートPCのディスプレイでは
文字を読むことが難しくなるため,それ以上の数の情報を把握したい場合は
情報を集約して表示する必要がある.


\section{日本語以外での『わかるらんど』の利用}

WISS2016での『わかるらんど』で投稿された558種類のスタンプのうち,
意味の通るテキストで作られたスタンプが341種類,
画像・絵文字・記号で作られたスタンプは217種類であった.
テキストスタンプがこれだけの数作られた理由は,
日本でのコンファレンスであったからであると思われる.
『わかるらんど』のスタンプは文字数が多くなればなるほど文字サイズが小さくなるため,
できる限り少ない文字数でスタンプを作ることが望ましい.
表意文字である漢字が使われる日本語は,少ない文字数で意味の通る言葉を作りやすく,
意味の通るテキストで作られたスタンプ341種類のうち158種類は4文字以下であった.

漢字圏ではないコンファレンスでの『わかるらんど』の運用は行っていないが,
漢字圏ではないコンファレンスではここまで多くのテキストスタンプは利用されず,
画像や絵文字によるスタンプが多く利用されると思われる.


\section{『わかるらんど』の情報ダッシュボードのレイアウト}

『わかるらんど』のダッシュボードは
単純な形状のセルを指定どおりに並べているだけであるが,
第\ref{chap:dashboard}章で紹介した
表示するべき情報を選択する手法\cite{Jones:2015:ECI:2800835.2800963}や,
セルの自動配置手法\cite{Hertzog:2015:BSP:2678025.2701383}は,
よりわかりやすい表示のための配置手法の検討は意義があると思われる.

会議での議論を促進するために第3章で紹介したようなチャットシステムは多数存在するが,
情報ダッシュボードのような型式で感情や意見など書き込んで
一覧できるチャットシステムは存在しない.
現在のリアルタイム情報共有がタイムライン表示が主流である中,
情報ダッシュボード形式の視覚化システムを模索することは,
インタラクション研究コミュニティにとって有意義であると考える.